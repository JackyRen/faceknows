% tex file
% AUTHOR:   rj
% FILE:     /media/files/projects/faceknows/doc/server/server-doc-v1.tex
% CREATED:  11:40:30 24/05/2013
% MODIFIED: 23:36:53 24/05/2013
\documentclass[a4paper]{article}
\usepackage{ctex}
\usepackage[top=1in, bottom=1in, left=1in, right=1in]{geometry}
\usepackage{mdwlist}


\begin{document}
\title{Server接口说明}
\author{任杰}
\maketitle
\newpage
\tableofcontents
\newpage

\section{概述}
这组JSP主要实现了对SDK的一些功能的封装
\section{信息管理部分}
\subsection{person}
\subsubsection{addFace.jsp}
\begin{itemize}
	\item{\bf description}
		向一个人的账户中加入若干张脸,加入脸之后会对这个人所在的所有组train一遍
	\item{\bf Input}
		\begin{basedescript}{\desclabelstyle{\pushlabel}\desclabelwidth{6em}}
		\item[faceid] 脸的id,face++系统给的,多个用逗号(,)隔开
		\item[personid] 人的ID 
		\end{basedescript}
	\item{\bf Output}
		nothing
	\item{\bf address} ./person/addFace.jsp
\end{itemize}
\subsubsection{create.jsp}
\begin{itemize}
	\item{\bf description}
		创建一个人
	\item{\bf Input}
		\begin{basedescript}{\desclabelstyle{\pushlabel}\desclabelwidth{6em}}
		\item[personid] 人的ID 
		\item[faceid(optional)] 脸的id,多个用逗号(,)隔开,会加入到该人里
		\item[groupid(optional)] 该人要加入的组,多个用逗号(,)隔开。
		\end{basedescript}
	\item{\bf Output}
		nothing
	\item{\bf address} ./person/create.jsp
\end{itemize}
\subsubsection{delete.jsp}
\begin{itemize}
	\item{\bf description}
		删除一个人
	\item{\bf Input}
		\begin{basedescript}{\desclabelstyle{\pushlabel}\desclabelwidth{6em}}
		\item[personid] 人的ID 
		\end{basedescript}
	\item{\bf Output}
		nothing
	\item{\bf address} ./person/delete.jsp
\end{itemize}


\subsection{group}
\subsubsection{addPerson.jsp}
\begin{itemize}
	\item{\bf description}
		向一个group中加入若干个人,加入之后会对group进行训练
	\item{\bf Input}
		\begin{basedescript}{\desclabelstyle{\pushlabel}\desclabelwidth{6em}}
		\item[personid] 人的ID , 多个用逗号隔开
		\item[groupid] 进行操作的组的id
		\end{basedescript}
	\item{\bf Output}
		nothing
	\item{\bf address} ./group/addPerson.jsp
\end{itemize}
\subsubsection{create.jsp}
\begin{itemize}
	\item{\bf description}
		创建一个Group
	\item{\bf Input}
		\begin{basedescript}{\desclabelstyle{\pushlabel}\desclabelwidth{6em}}
		\item[groupid] 组的ID, 我们自己的系统确定
		\item[personid(optional)] 该组要加入的人的ID,多个用逗号(,)隔开。
		\end{basedescript}
	\item{\bf Output}
		nothing
	\item{\bf address} ./group/create.jsp
\end{itemize}
\subsubsection{delete.jsp}
\begin{itemize}
	\item{\bf description}
		删除一个Group
	\item{\bf Input}
		\begin{basedescript}{\desclabelstyle{\pushlabel}\desclabelwidth{6em}}
		\item[groupid] 组的ID 
		\end{basedescript}
	\item{\bf Output}
		nothing
	\item{\bf address} ./group/delete.jsp
\end{itemize}
\subsubsection{removePerson.jsp}
\begin{itemize}
	\item{\bf description}
		从一个Group中移除一些人
	\item{\bf Input}
		\begin{basedescript}{\desclabelstyle{\pushlabel}\desclabelwidth{6em}}
		\item[personid] 人的ID , 多个用逗号隔开
		\item[groupid] 进行操作的组的id
		\end{basedescript}
	\item{\bf Output}
		nothing
	\item{\bf address} ./group/removePerson.jsp
\end{itemize}
\section{训练部分}
\subsection{detectImage.jsp}
\begin{itemize}
	\item{\bf description}
		检测一张图片中的人脸
	\item{\bf Input}
		\begin{basedescript}{\desclabelstyle{\pushlabel}\desclabelwidth{6em}}
		\item[url] 图片的url
		\end{basedescript}
	\item{\bf Output}
		JSON格式的检测信息
	\item{\bf address} ./detectImage.jsp
	\item{\bf memo}
		nothing
\end{itemize}

\section{检测部分}
\subsection{identify.jsp}
\begin{itemize}
	\item{\bf description}
		检测照片中的人脸和某个组中最相似的人
	\item{\bf Input}
		\begin{basedescript}{\desclabelstyle{\pushlabel}\desclabelwidth{6em}}
		\item[groupid] 我们内部的组的id
		\item[url] 图片的url
		\end{basedescript}
	\item{\bf Output}
		JSON格式的检测信息
	\item{\bf address} ./identify.jsp
	\item{\bf memo}
		原计划中我要在这里对返回值简化,但写的过程中感觉有点麻烦,涉及对JSON数据的修改,face++的sdk只支持读取,要修改可能又要引入一些包,放到客户端做就只需要读取了。
\end{itemize}


\end{document}


